\begin{problem}{\rl{یخدارچی}}
%{\textsl{standard input}}{\textsl{standard output}}
{}{}
{\rl{۱ ثانیه}}{\rl{۱۰۰ مگابایت}}{}
    محمدجواد که پشت‌کار بالایی دارد، در محمدجواد که پشت‌کار بالایی دارد، در ابتدا یک آبدارچی بود (البته حالا او با پشت‌کارش تمامی پله‌های ترقی را طی کرده‌است). او باید در هر لحظه می‌دانست که وضعیت آب داخل سماور چگونه است. برای همین او یک دماسنج به سماور وصل کرد و باید با توجه به دمای آب داخل سماور وضعیت آب را می‌سنجید. می‌دانیم در فشار معمولی آب در دمای بیش از ۱۰۰ درجه بخار است و در دمای کمتر از ۰ یخ می‌زند.(ممکن است در زمستان آب داخل سماور یخ بزند)
    حالا شما می‌دانید برای موفقیت باید پشت‌کار داشته باشید. برای همین دیر یا زود باید تصمیم بگیرید که راه محمدجواد را ادامه دهید. برای این کار شما باید با توجه به دمای داخل سماور وضعیت آب (در واقع $H_2O$) داخل سماور را مشخص کنید.

    \InputFile
    در سطر اول ورودی یک عدد صحیح $T$ آمده‌است که نشان‌دهنده‌ی دمای آب داخل سماور است.

    $$ -273 < T \le 6\ 000$$



    \OutputFile

    اگر دمای داخل سماور بیش از ۱۰۰ درجه بود، "Steam" چاپ شود. اگر دمای آن زیر ۰ بود، "Ice" و در غیر این صورت، "Water" چاپ شود.
    \Examples

    \Example
    \begin{example}
        \exmp{
        \importproblemtestcase{ice-maker}{in}{1}
        }{%
        \importproblemtestcase{ice-maker}{out}{1}
        }
        %
    \end{example}

    \Example
    \begin{example}
        \exmp{
        \importproblemtestcase{ice-maker}{in}{2}
        }{%
        \importproblemtestcase{ice-maker}{out}{2}
        }
        %
    \end{example}

    \Example
    \begin{example}
        \exmp{
        \importproblemtestcase{ice-maker}{in}{3}
        }{%
        \importproblemtestcase{ice-maker}{out}{3}
        }
        %
    \end{example}


\end{problem}
