\gdef\thisproblemauthor{\rl{سپهر هاشمی}}
\begin{problem}{\rl{کپی پیست!}}
%{\textsl{standard input}}{\textsl{standard output}}
{}{}
{\rl{۱ ثانیه}}{\rl{۲۵۶ مگابایت}}{}
دستیاران آموزشی کلاس دکتر اعتمادی تمرین اول تمام بچه‌ها را دریافت کرده و می‌خواهند لیست دانشجوهای کلاس را به روزرسانی کنند. این کار به این صورت انجام می‌شود که کسانی که ۵۰ درصد نمره‌ی کلِ تمرین‌های سری اول را نگرفته‌اند یا تمرین را کپ زده‌اند (حتی ۱ سوال) از لیست کلاسی حذف می‌شوند. هم‌چنین لیست نمرات بر اساس شماره دانشجویی از کوچک به بزرگ باید مرتب باشد.
    \InputFile
 ورودی شامل $n + 1$ خط است که در خط اول ورودی تعداد دانشجویان  $n$ آمده است  و در $n$ خط بعدی در هر خط شماره دانشجوییِ دانشجو و نمره‌ی سوال‌های ۱ و ۲ و ۳ به ترتیب آمده است و از 
 \colorbox{gray!10}{''\lr{\texttt{,}}``}
 به عنوان جداکننده استفاده شده است.
 اگر تمرینی کپ زده شده باشد بجای نمره‌ی آن عبارت
 \colorbox{gray!10}{\lr{\texttt{copy}}}
  نوشته شده است.

    $$ 1 \le n \le 100 $$

    \OutputFile

در هر خط شماره دانشجویی دانشجویان باقی‌مانده به ترتیب شماره‌ی دانشجویی از کوچک به بزرگ آمده است.
\Examples

\insertQueraTestCaseSampleInOutTable{copy-paste}{1}
\insertQueraTestCaseSampleInOutTable{copy-paste}{2}


\end{problem}
