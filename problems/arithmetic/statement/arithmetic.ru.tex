\gdef\thisproblemauthor{Иван Казменко}
\gdef\thisproblemdeveloper{Иван Казменко}
\gdef\thisproblemorigin{\texttt{XXXIII} Чемпионат СПбГУ}
\begin{problem}{Арифметика на доске}
{arithmetic.in}{arithmetic.out}
{2 секунды (\textsl{3 секунды для Java})}{256 мебибайт}{}
\
لورم ایپسوم متن ساختگی با تولید سادگی نامفهوم از صنعت چاپ و با استفاده از طراحان گرافیک است. چاپگرها و متون بلکه روزنامه و مجله در ستون و سطرآنچنان که لازم است و برای شرایط فعلی تکنولوژی مورد نیاز و کاربردهای متنوع با هدف بهبود ابزارهای کاربردی می باشد. کتابهای زیادی در شصت و سه درصد گذشته، حال و آینده شناخت فراوان جامعه و متخصصان را می طلبد تا با نرم افزارها شناخت بیشتری را برای طراحان رایانه ای علی الخصوص طراحان خلاقی و فرهنگ پیشرو در زبان فارسی ایجاد کرد. در این صورت می توان امید داشت که تمام و دشواری موجود در ارائه راهکارها و شرایط سخت تایپ به پایان رسد وزمان مورد نیاز شامل حروفچینی دستاوردهای اصلی و جوابگوی سوالات پیوسته اهل دنیای موجود طراحی اساسا مورد استفاده قرار گیرد.
\InputFile

در سطر اول ورودی، عدد $n$ آمده‌است.

سپس در $i$ امین سطر از $n$ سطر بعدی، دو عدد $r_i$ و $b_i$ آمده‌است که به ترتیب تعداد افرادیست که در شهر $i$ ام به *قرمزگرا*ها و *آبی‌طلب* ها رای میدهند.

$$0 \le r_i, b_i \le 5000$$

\OutputFile

در خروجی باید بیشترین درصد کنگره را که مهدیس میتواند بدست بیاورد تا دورقم اعشار چاپ شود.


\Examples

\begin{example}
\exmp{
2
1 2
}{%
1
}%
\exmp{
3
1 2 3
}{%
0
}%
\exmp{
4
16 2 3 4
}{%
2
}%
\end{example}

\Explanations

В первом примере можно заменить $1$ и $2$ на $|2 - 1| = 1$.

Во втором примере можно сначала заменить $1$ и $2$ на $1 + 2 = 3$,
а затем получить $|3 - 3| = 0$.

В третьем примере двойку можно получить, например, так:
$2 + 4 = 6$, $3 \cdot 6 = 18$ и $|16 - 18| = 2$.

\end{problem}
