%\gdef\thisproblemauthor{\rl{نام صاحب اثر اینجا قرار می‌گیرد}}
%\gdef\thisproblemdeveloper{\rl{نام توسعه‌دهنده اینجا قرار می‌گیرد}}
\begin{problem}{\rl{آخ جون طرف نیست!}}
{standard input}{standard output}
{\rl{۰.۵ ثانیه}}{\rl{۲۵۶ مگابایت}}{}

    مهدی که از کدزدن خسته شده‌است، دوست ندارد که در سوال‌هایی که در آن‌ها نیاز به کد زدن هست، از نام او استفاده شود. چندی پیش مهدی متوجه شد که پویان (که یک نوجوان تپل است) و دو نفر از دوستانش به دور از چشم او تعدادی صورت سوال برای مسابقه‌ای در Quera نوشته‌اند و از او بعنوان شخصیت اصلی داستان‌های آن استفاده کرده‌اند. این با معیارهای مهدی جور در نمی‌آید؛ پس او تمام تلاشش را می‌کند که صورت آن سوال‌ها را تغییر دهد.

    این سه نفر، پویان و دو نفر از دوستانش، هر یک در تعدادی از روزهای هفته به Quera می‌روند. مهدی در آغاز هفته روزهایی از هفته که هریک از این سه تن در Quera هستند را از آن‌ها پرسیده است. او می‌خواهد بداند چند روز در هفته می‌تواند به Quera برود که هیچ‌یک از این افراد در آنجا نباشند تا بتواند صورت سوال‌ها را به دلخواه خودش تغییر دهد. حال مهدی برنامه‌ی روزهایی که این سه نفر به Quera می‌روند را به شما می‌دهد و شما بگویید که مهدی چند روز در این هفته می‌تواند به اصلاح این صورت سوال‌ها بپردازد.
    \Example

    \begin{examplethree}
        \exmp{%
        4
        shanbe 1shanbe 2shanbe 3shanbe
        1
        5shanbe
        3
        1shanbe 3shanbe 5shanbe
        }{
        2
        }{%
        \vskip 0pt
        }
        %
        \exmp{%
        2
        shanbe 2shanbe
        2
        1shanbe 3shanbe
        3
        jome 5shanbe 4shanbe
        }{
        0
        }{%
        \vskip 0pt
        }
        %
    \end{examplethree}

    \Explanation

    در ورودی روز‌هایی از هفته که هریک از این افراد به Quera می‌روند در این قالب آمده‌است:

    توصیف روز‌های هر یک از این سه فرد در دو سطر آمده است. (پس در مجموع ورودی شامل ۶ سطر می‌شود.) در سطر اول هر توصیف تعداد روز‌هایی که این فرد در هفته به Quera می‌رود آمده‌است و سپس در سطر بعدی آن، نام روز‌هایی که آن فرد به Quera می‌رود آمده‌است. تضمین می‌شود که تعداد این روز‌ها در سطر اول هر توصیف، با تعداد نام روزها در سطر دوم آن برابر است و نام یک روز از هفته در یک توصیف حداکثر یک‌ بار آمده‌است. همچنین تضمین می‌شود هریک از این افراد در این هفته حداقل یک روز به Quera می‌روند.

    نام روز‌های هفته:
\begin{itemize}
    \item shanbe
    \item 1shanbe
    \item 2shanbe
    \item 3shanbe
    \item 4shanbe
    \item 5shanbe
    \item jome
\end{itemize}

\end{problem}
